\documentclass[fontsize=12pt, paper=book, DIV=calc]{scrbook}

% PACKAGES
\usepackage[top=35mm, bottom=70mm, inner=20mm, outer=30mm]{geometry}
\usepackage[english]{babel}
\usepackage[utf8]{inputenc}
\usepackage[T1]{fontenc}
\usepackage{graphicx, tabularx}
\usepackage{physics, mathtools}
\usepackage{amssymb, enumitem}
\usepackage[colorlinks=true, linkcolor=myblue]{hyperref}
\usepackage{caption, subcaption,  xcolor}
\usepackage{cleveref}
\usepackage{biblatex}

\definecolor{myblue}{RGB}{0,82,155}
% Formatting settings
\KOMAoptions{headsepline=true}
\setlength{\headsep}{1.5em} % Increase space between header and body
\usepackage{scrlayer-scrpage} % Ensure scrlayer-scrpage is loaded for customization
\addtokomafont{headsepline}{\color{myblue}} % Set headsepline color to blue
\setkomafont{disposition}{\color{myblue}\bfseries}
\setlength{\parskip}{1em}
\renewcommand{\headfont}{\color{myblue}\normalfont\slshape}


\hypersetup{colorlinks=true, linkcolor=myblue, citecolor=myblue, urlcolor=myblue}

\addbibresource{references.bib}  % Your bibliography file

\title{OCR Project Report}
\author{Claudia Wisniewski, Fausto Frisenna, Nikolina Karamarko, Kazem Zhour}
\date{\today}

\begin{document}

\maketitle

\tableofcontents
\newpage
Optical recognition is a field that consists of extracting the text out of an image in order for it to be manipulated by a computer. Even though paper material is a natural way of communication for humans, it is not understandable by a computer. For this reason, a lot of insurance companies focus on this type of tool in order to OCR (Optical Character Recognition) process handwritten documents (birth certificates, sale certificates)
The goal of this project is to use a Deep Learning algorithm (Computer Vision) to recognize the characters within PDF/PNG files.
\chapter{Report 1: introduce, describe, visualize and manipulate your dataset}

\section{Introduction to the project}
\subsection*{Context}
\begin{itemize}[label=\textbullet]
    \item Context of the project's integration into your \textbf{business}.
    \item From a \textbf{technical} point of view.
    \item From an \textbf{economic} point of view.
    \item From a \textbf{scientific} point of view.
\end{itemize}

\subsection*{Objectives}
\begin{itemize}[label=\textbullet]
    \item What are the main objectives to be achieved? Describe in a few lines.
    \item For each member of the group, specify the level of expertise around the problem addressed.
    \item Have you contacted business experts to refine the problem and the underlying models? If yes, detail the contribution of these interactions.
    \item (Are you aware of a similar project within your company or in your entourage? What is its progress? How has it helped you realize your project? How does your project contribute to improving it?)
\end{itemize}

\section{Understanding and Manipulation of Data}

\subsection*{Framework}
\begin{itemize}[label=\textbullet]
    \item Which set(s) of data(s) did you use to achieve the objectives of your project?
    \item Are these data freely available? If not, who owns the data?
    \item Describe the volume of your dataset.
\end{itemize}

\subsection*{Relevance}
\begin{itemize}[label=\textbullet]
    \item Which variables seem most relevant to you with regard to your objectives?
    \item What is the target variable?
    \item What features of your dataset can you highlight?
    \item Are you limited by some of your data?
\end{itemize}

\subsection*{Pre-processing and Feature Engineering}
\begin{itemize}[label=\textbullet]
    \item Did you have to clean and process the data? If yes, describe your treatment process.
    \item Did you have to carry out normalization/standardization type transformations of your data? If yes, why?
    \item Are you considering dimension reduction techniques in the modeling part? If yes, why?
\end{itemize}

\subsection*{Visualizations and Statistics}
\begin{itemize}[label=\textbullet]
    \item Have you identified relationships between different variables? Between explanatory variables? And between your explanatory variables and the target(s)?
    \item Describe the distribution of these data, distribution, and outliers (pre/post-processing if necessary).
    \item Present the statistical analyses used to confirm the information present on the graphs.
    \item Conclude the elements noted above, allowing them to project themselves into the modeling part.
\end{itemize}
\printbibliography

\end{document}
